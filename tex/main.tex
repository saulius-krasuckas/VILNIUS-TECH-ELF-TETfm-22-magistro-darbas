% SPDX-License-Identifier: BlueOak-1.0.0
% SPDX-FileCopyrightText: 2024 Saulius Krasuckas <saulius.krasuckas_at_stud_vilniustech_lt>
% SPDX-FileCopyrightText: 2015,2017 Tomas Rekašius <tomas.rekasius_at_vgtu_lt>


\documentclass[12pt]{article}

% Via: https://github.com/trekasius/bakalaurinio-darbo-sablonas/blob/5ec479d16053077cf264a718ff032b56edf2bcdc/bd-sablonas.tex
% ------------------------------------------------------------------------------
%
%        VGTU ELF Kompiuterijos ir ryšių technologijų katedros (KRTK) magistro darbas
%
%        Šablono autorius: Tomas Rekašius
%          Darbo autorius: Saulius Krasuckas
%                 Versija: 0.0
%
%

% ------------------------------------------------------------------------------
%  PREAMBULĖ
% ------------------------------------------------------------------------------


%\usepackage[utf8]{inputenc}           % naudojama, kai .tex failas UTF-8 koduotės
                                       % nuo TeX Live 2018 versijos UTF-8 naudojama pagal nutylėjimą.
                                       % Via: https://www.overleaf.com/learn/latex/International_language_support#Font_encoding

%\usepackage[L7x]{fontenc}             % nurodoma lietuviško teksto koduotė Latin-7
%\usepackage{lmodern}                  % dokumente naudojamas šriftas Latin Modern
%\usepackage[lithuanian]{babel}        % nurodomi lietuviškos rašybos aspektai
                                       % Via: https://www.pvv.ntnu.no/~berland/latex/docs/babel.pdf#page=6

 \usepackage[T1]{fontenc}              % nurodoma teksto koduotė T1, dar žinoma kaip EC arba Cork % Via: https://en.wikipedia.org/wiki/Cork_encoding
 \usepackage{times}                    % dokumente naudojamas šriftas Times New Roman (nebenaudotinas)
%\usepackage{mathptmx}
%\usepackage{mathdesign}
%\usepackage{newtxtext}
%\usepackage{newtxmath}
%\usepackage{charter}

\usepackage{blindtext}                 % dokumento užpildymas pagal "Lorem Ipsum" šabloną
                                       % Via: https://texblog.org/2011/02/26/generating-dummy-textblindtext-with-latex-for-testing/

\usepackage{geometry}                  % paraščių ir kitų lapo parametrų nustatymai
\usepackage{microtype}                 % optimizuojami atstumai tarp raidžių žodyje ir tarp žodžių eilutėje
\usepackage{printlen}

\geometry{                             % paketo "geometry" nustatymai pagal VILNIUS TECH metodiką:
  a4paper,                             % - lapo dydis
     left = 30 mm,                     % - paraštės
    right = 10 mm,                     %   .
      top = 20 mm,                     %   .
   bottom = 20 mm,                     %   .
}                                      % Via: https://vilniustech.lt/files/5184/259/12/13_0/Baigiamųjų darbų metodikos nurodymai.pdf#page=84


% PAPILDOMI PAKETAI IR NUSTATYMAI ----------------------------------------------

\linespread{1.44}                      % nustatomas 1,5 dydžio tarpas tarp eilučių pagal Microsoft Word


% ------------------------------------------------------------------------------
%  DOKUMENTO PRADŽIA
% ------------------------------------------------------------------------------

\begin{document}

% ------------------------------------------------------------------ VIRŠELIS --

\newgeometry{
     left = 10 mm,                     % Suvienodinta su "right" pilnam antraštinių lapų sucentravimui
    right = 10 mm,                     %   .
      top = 20 mm,                     %   .
   bottom = 20 mm,                     %   .
}

\begin{titlepage}
\centering

   %\printlength{\baselineskip}        % sugeneruoja: 20.88004pt

% Small Caps: visos raidės virsta didžiosiomis, tačiau pradinės mažosios tampa šiek tiek mažesnėmis nei pradinės didžiosios
{\scshape
    {\Large Vilniaus Gedimino Technikos Universitetas}    \\[1.30pt]
    {\large Elektronikos Fakultetas}                      \\[0.20pt]
    {\large Kompiuterijos ir Ryšių Technologijų Katedra}  \\[106.0pt]
}

    {\large Saulius Krasuckas}                            \\[24.5pt]

    {\large{\bfseries{MPTCP TYRIMAS DUOMENŲ CENTRUOSE}}}  \\[-5.0pt]
    {\large{EVALUATION OF MULTIPATH TCP PROTOCOL
                USE IN DATA CENTER NETWORKS}}             \\[36.5pt]

    {\large Baigiamasis magistro darbas}

    Telekomunikacijų inžinerijos studijų programa,
    valstybinis kodas 6211EX052                           \\
    Telekomunikacijų technologijų specializacija          \\
    Elektronikos ir elektros inžinerijos studijų kryptis  \\

    \vspace{\fill}

\end{titlepage}

\restoregeometry                       % Grąžinta pradinė, universiteto rekomenduojama geometrija

  \blindtext[10]                       % Dešimt "Lorem Ipsum" pastraipų, kad užimtų daugiau nei vieną puslapį
\end{document}

% Pradinė struktūra pagal https://latex-tutorial.com/tutorials/first-document/
