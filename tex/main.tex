% SPDX-License-Identifier: BlueOak-1.0.0
% SPDX-FileCopyrightText: 2024 Saulius Krasuckas <saulius.krasuckas_at_stud_vilniustech_lt>
% SPDX-FileCopyrightText: 2015,2017 Tomas Rekašius <tomas.rekasius_at_vgtu_lt>


\documentclass[12pt]{article}

% Via: https://github.com/trekasius/bakalaurinio-darbo-sablonas/blob/5ec479d16053077cf264a718ff032b56edf2bcdc/bd-sablonas.tex
% ------------------------------------------------------------------------------
%
%        VGTU ELF Kompiuterijos ir ryšių technologijų katedros (KRTK) magistro darbas
%
%        Šablono autorius: Tomas Rekašius
%          Darbo autorius: Saulius Krasuckas
%                 Versija: 0.0
%
%

% ------------------------------------------------------------------------------
%  PREAMBULĖ
% ------------------------------------------------------------------------------


%\usepackage[utf8]{inputenc}           % naudojama, kai .tex failas UTF-8 koduotės
                                       % nuo TeX Live 2018 versijos UTF-8 naudojama pagal nutylėjimą.
                                       % Via: https://www.overleaf.com/learn/latex/International_language_support#Font_encoding

%\usepackage[L7x]{fontenc}             % nurodoma lietuviško teksto koduotė Latin-7
%\usepackage{lmodern}                  % dokumente naudojamas šriftas Latin Modern
 \usepackage[lithuanian]{babel}        % nurodomi lietuviškos rašybos aspektai
                                       % Via: https://www.pvv.ntnu.no/~berland/latex/docs/babel.pdf#page=6

 \usepackage[T1]{fontenc}              % nurodoma teksto koduotė T1, dar žinoma kaip EC arba Cork % Via: https://en.wikipedia.org/wiki/Cork_encoding
 \usepackage{times}                    % dokumente naudojamas šriftas Times New Roman (nebenaudotinas)
%\usepackage{mathptmx}
%\usepackage{mathdesign}
%\usepackage{newtxtext}
%\usepackage{newtxmath}
%\usepackage{charter}

 \usepackage{blindtext}                % dokumento užpildymas pagal "Lorem Ipsum" šabloną
                                       % Via: https://texblog.org/2011/02/26/generating-dummy-textblindtext-with-latex-for-testing/

 \usepackage{geometry}                 % paraščių ir kitų lapo parametrų nustatymai
 \usepackage{microtype}                % optimizuojami atstumai tarp raidžių žodyje ir tarp žodžių eilutėje
 \usepackage{printlen}                 % tiksliam tarpų tarp eilučių pamatavimui
 \usepackage{graphicx}                 % grafinių failų įterpimas ir kiti nustatymai
 \usepackage{indentfirst}              % atitraukiama pirmoji naujo skyriaus eilutė
 \usepackage{titlesec}                 % leidžia keisti skyriaus pavadinimo stilių


\geometry{                             % paketo "geometry" nustatymai pagal VILNIUS TECH metodiką:
  a4paper,                             % - lapo dydis
     left = 30 mm,                     % - paraštės
    right = 10 mm,                     %   .
      top = 20 mm,                     %   .
   bottom = 20 mm,                     %   .
}                                      % Via: https://vilniustech.lt/files/5184/259/12/13_0/Baigiamųjų darbų metodikos nurodymai.pdf#page=84


% PAPILDOMI PAKETAI IR NUSTATYMAI ----------------------------------------------

 \graphicspath{{tex/imgs}}             % nustatome kelią iki paveikslėlių katalogo
 \linespread{1.44}                     % nustatomas 1,5 dydžio tarpas tarp eilučių pagal Microsoft Word
 \setlength{\parindent}{20mm}          % įtraukos dydis (7..20 mm)
                                       % Via: ---"---
                                       % Via: https://www.overleaf.com/learn/latex/Articles/How_to_change_paragraph_spacing_in_LaTeX#.5Cparindent_.28TeX_primitive.29
 \titlelabel{\thetitle.\hspace{2mm}}   % dedamas taškas po skyriaus numeriu tekste,
                                       % taip pat sumažinamas tarpelis nuo \quad (1em) iki 2mm

% ------------------------------------------------------------------------------
%  DOKUMENTO PRADŽIA
% ------------------------------------------------------------------------------

\begin{document}

\newgeometry{
     left = 10 mm,                     % Suvienodinta su "right" pilnam antraštinių lapų sucentravimui
    right = 10 mm,                     %   .
      top = 20 mm,                     %   .
   bottom = 20 mm,                     %   .
}

% ------------------------------------------------ VIRŠELIS / TITULINIS LAPAS --

\begin{titlepage}
\centering

    \vspace*{-9.5pt}
    \includegraphics[scale = 0.105]{Herbas.png}           \\[3.9pt]

% Small Caps: visos raidės virsta didžiosiomis, tačiau pradinės mažosios tampa šiek tiek mažesnėmis nei pradinės didžiosios
{\scshape
    {\Large Vilniaus Gedimino Technikos Universitetas}    \\[1.30pt]
    {\large Elektronikos Fakultetas}                      \\[0.20pt]
    {\large Kompiuterijos ir Ryšių Technologijų Katedra}  \\[106.0pt]
}

   % TODO: iškelti tolimesnę \large seką į bendrą \large bloką
   % (ir perderinti dėl to vėl pakitusius tarp tarp eilučių)

    {\large Saulius Krasuckas}
   %{\large Vardas Pavardė}
                                                          \\[24.5pt]
    {\large{\bfseries{MPTCP TYRIMAS DUOMENŲ CENTRUOSE}}}
   %{\large{\bfseries BAIGIAMOJO DARBO PAVADINIMAS
   %                                   (LIETUVIŲ KLB.)}}
                                                          \\[-5.0pt]
    {\large{EVALUATION OF MULTIPATH TCP PROTOCOL USE      \\
                   IN DATA CENTER NETWORKS}}
   %{\large{\bfseries BD PAVADINIMAS (ANGLŲ KLB.)}}
                                                          \\[36.5pt]

    {\large Baigiamasis magistro darbas}
                                                          \\[1.9pt]
    Telekomunikacijų inžinerijos studijų programa,
    valstybinis kodas 6211EX052                           \\
    Telekomunikacijų technologijų specializacija          \\
    Elektronikos ir elektros inžinerijos studijų kryptis  \\

    \vspace{\fill}

    Vilnius, \the\year

\end{titlepage}

% --------------------------------------------------------- ANTRAŠTINIS LAPAS --

\begin{titlepage}
\centering

    {\Large Antraštinis lapas}

\end{titlepage}
\restoregeometry                       % Grąžinta pradinė, universiteto rekomenduojama geometrija
\newpage

% ------------------------------------------------------------ DARBO UŽDUOTIS --

\begin{titlepage}
\centering

    {\Large Darbo užduotis}

\end{titlepage}
\newpage

% ------------------------------------------------------ ANOTACIJA (LIET. K.) --

\begin{titlepage}
\centering

    {\Large Anotacija lietuvių kalba}

\end{titlepage}
\newpage

% ----------------------------------------------------- ANOTACIJA (ANGLŲ. K.) --

\begin{titlepage}
\centering

    {\Large Anotacija anglų kalba}

\end{titlepage}
\newpage

% --------------------------------------------------- SĄŽININGUMO DEKLARACIJA --

\begin{titlepage}
\centering

    {\Large Baigiamojo darbo sąžiningumo deklaracija}

\end{titlepage}
\newpage

% ------------------------------------------------------------------- TURINYS --

    \tableofcontents

\newpage

% --------------------------------------------------------- PAVEIKSLŲ SĄRAŠAS --

\section*{Paveikslų sąrašas}


\newpage

% --------------------------------------------------------- LENTELLIŲ SĄRAŠAS --

\section*{Lentelių sąrašas}


\newpage

% ---------------------------------------------------------------- SANTRUMPOS --

\section*{Santrumpos}


\newpage

% -------------------------------------------------------------------- ĮVADAS --

\section*{Įvadas}
\addcontentsline{toc}{section}{Įvadas}

   %\printlength{\baselineskip}        % sugeneruoja: 20.88004pt
   %\printlength{\parindent}           % sugeneruoja: 17.62482pt
   %\printlength{\quad}                % sugeneruoja: 01em

    \blindtext                         % Santrauka iš "Lorem Ipsum"
    \par                               % Nauja pastraipa
                                       % Via: https://www.overleaf.com/learn/latex/Paragraphs_and_new_lines#Starting_a_new_paragraph
    \blindtext                         % Santrauka iš "Lorem Ipsum"

\newpage

% ------------------------------------------------------ LITERATŪROS APŽVALGA --

\section{Literatūros apžvalga}


\newpage

% --------------------------------------------------------- TEORINIAI TYRIMAI --

\section{Teoriniai tyrimai}

    \blindtext[10]                     % Dešimt "Lorem Ipsum" pastraipų, kad užimtų daugiau nei vieną puslapį

\newpage

% -------------------------------------------------- EKSPERIMENTINIAI TYRIMAI --

\section{Eksperimentiniai tyrimai}


\newpage

% ----------------------------------------------------- REZULTATŲ PALYGINIMAS --

\section{Teorinių ir eksperimentinių rezultatų palyginimas}


\newpage

% ----------------------------------------------------- IŠVADOS IR PASIŪLYMAI --

\section*{Išvados ir pasiūlymai}
\addcontentsline{toc}{section}{Išvados ir pasiūlymai}


\newpage

% ------------------------------------------------------- LITERATŪROS SĄRAŠAS --

\section*{Literatūros sąrašas}
\addcontentsline{toc}{section}{Literatūros sąrašas}


\newpage

% ------------------------------------------------------------------- PRIEDAI --

\section*{PRIEDAI}
\addcontentsline{toc}{section}{PRIEDAI}


\newpage

\end{document}

% Pradinė struktūra pagal https://latex-tutorial.com/tutorials/first-document/
